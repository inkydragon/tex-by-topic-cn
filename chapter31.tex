% -*- coding: utf-8 -*-
\documentclass{book}

\input{preamble}
\setcounter{chapter}{30}

\begin{document}

%\chapter{Allocation}\label{alloc}
\chapter{寄存器分配}\label{alloc}

%\TeX\ has registers of a number of types. For some of these,
%explicit commands exist to define a synonym for a certain register;
%for all of them macros exist in the plain format
%to allocate an unused register. This chapter treats
%the synonym and allocation commands, and discusses
%some guidelines for macro writers regarding allocation.
\TeX\ 的寄存器是数字类型的。对于其中的一些寄存器,
我们可以使用显式的命令定义某个特定寄存器的别名;
在 plain \TeX\ 中,
所有类型的寄存器都可以通过对应的宏来分配一个未使用的寄存器。
本章主要关注别名定义和寄存器分配命令。
对于宏编写者,我们讨论了一些关于寄存器分配的方针。

%\begin{inventory}
%\item [\cs{countdef}] 
%      Define a synonym for a \cs{count} register.
%\item [\cs{dimendef}]
%      Define a synonym for a \cs{dimen} register.
%\item [\cs{muskipdef}]
%      Define a synonym for a \cs{muskip} register.
%\item [\cs{skipdef}] 
%      Define a synonym for a \cs{skip} register.
%\item [\cs{toksdef}] 
%      Define a synonym for a \cs{toks} register.
%\item [\cs{newbox}]
%      Allocate an unused \cs{box} register.
%\item [\cs{newcount}]
%      Allocate an unused \cs{count} register.
%\item [\cs{newdimen}]
%      Allocate an unused \cs{dimen} register.
%\item [\cs{newfam}]
%      Allocate an unused math family.
%\item [\cs{newinsert}]
%      Allocate an unused insertion class.
%\item [\cs{newlanguage}]
%      (\TeX3 only)
%      Allocate a new language number.
%\item [\cs{newmuskip}]
%      Allocate an unused \cs{muskip} register.
%\item [\cs{newskip}]
%      Allocate an unused \cs{skip} register.
%\item [\cs{newtoks}]
%      Allocate an unused \cs{toks} register.
%\item [\cs{newread}]
%      Allocate an unused input stream.
%\item [\cs{newwrite}]
%      Allocate an unused output stream.
%\end{inventory}
\begin{inventory}
\item [\cs{countdef}] 
      定义了一个 \cs{count} 寄存器的别名。
\item [\cs{dimendef}]
      定义了一个 \cs{dimen} 寄存器的别名。
\item [\cs{muskipdef}]
      定义了一个 \cs{muskip} 寄存器的别名。
\item [\cs{skipdef}] 
      定义了一个 \cs{skip} 寄存器的别名。
\item [\cs{toksdef}] 
      定义了一个 \cs{toks} 寄存器的别名。
\item [\cs{newbox}]
      分配了一个未使用的 \cs{box} 分寄存器。
\item [\cs{newcount}]
      分配了一个未使用的 \cs{count} 分寄存器。
\item [\cs{newdimen}]
      分配了一个未使用的 \cs{dimen} 分寄存器。
\item [\cs{newfam}]
      分配了一个未使用的数学字体族。
\item [\cs{newinsert}]
      分配了一个未使用的插入类。
\item [\cs{newlanguage}]
      (仅 \TeX3)
      分配了一个新的语言数字。
\item [\cs{newmuskip}]
      分配了一个未使用的 \cs{muskip} 分寄存器。
\item [\cs{newskip}]
      分配了一个未使用的 \cs{skip} 分寄存器。
\item [\cs{newtoks}]
      分配了一个未使用的 \cs{toks} 分寄存器。
\item [\cs{newread}]
      分配了一个未使用的输入流。
\item [\cs{newwrite}]
      分配了一个未使用的输出流。
\end{inventory}

%%\point Allocation commands
%\section{Allocation commands}
%\index{registers!allocation of|(}
%\point Allocation commands
\section{寄存器分配命令}
\index{registers!allocation of|(}

%In plain \TeX, \cs{new...} macros are defined for
%allocation of registers.
%The registers of \TeX\ fall into two classes that are 
%allocated in different ways. This is treated below.
在 plain \TeX 中, 定义了宏 \cs{new...} 用于分配寄存器。
\TeX\ 的寄存器可以根据分配方式的不同而分为两类。这一点在下面会谈到。

%The \csidx{newlanguage} macro of plain \TeX\ 
%does not allocate any register. Instead it merely assigns
%a number, starting from~0.
%\TeX\ (version~3) can have at most 256 different
%sets of hyphenation patterns.
plain \TeX\ 的 \csidx{newlanguage} 宏并不会分配任何的寄存器。
相反的,它只是从0开始,分配了一个数字。
\TeX3 最多有256种断字组合模式。

%The \cs{new...} macros of plain \TeX\ are defined to be
%\cs{outer} (see Chapter~\ref{macro} for a precise explanation),
%which precludes use of the allocation macros in other macros.
%Therefore the \LaTeX\ format redefines these macros
%without the \cs{outer} prefix.
plain \TeX\ 的 \cs{new...} 宏被定义为 \cs{outer}
(精确定义详见 Chapter~\ref{macro}),
\cs{outer} 避免了在其他宏中使用分配宏。
因此,\LaTeX\ 在重定义这些宏时没有带上 \cs{outer} 前缀。

%%\spoint \cs{count}, \cs{dimen}, \cs{skip}, \cs{muskip}, \cs{toks}
%\subsection{\cs{count}, \cs{dimen}, \cs{skip}, \cs{muskip}, \cs{toks}}
%\spoint \cs{count}, \cs{dimen}, \cs{skip}, \cs{muskip}, \cs{toks}
\subsection{\cs{count}, \cs{dimen}, \cs{skip}, \cs{muskip}, \cs{toks}}

%For these registers there exists a \gr{registerdef} command,
%for instance \cs{countdef}, to couple a specific register
%to a control sequence:
%\begin{Disp}\gr{registerdef}\gr{control 
%    sequence}\gr{equals}\gr{8-bit number}\end{Disp}
对于这些寄存器,存在一个 \gr{registerdef} 命令。
举个例子:\cs{countdef}, 将寄存器与特定的控制序列组合在一起:
\begin{Disp}\gr{registerdef}\gr{control 
    sequence}\gr{equals}\gr{8-bit number}\end{Disp}

%After the definition
%\begin{verbatim}
%\countdef\MyCount=42
%\end{verbatim}
%the allocated register can be used as
%\begin{verbatim}
%\MyCount=314
%\end{verbatim}
%or
%\begin{verbatim}
%\vskip\MyCount\baselineskip
%\end{verbatim}
在定义之后
\begin{verbatim}
\countdef\MyCount=42
\end{verbatim}
分配的寄存器可以这样使用:
\begin{verbatim}
\MyCount=314
\end{verbatim}
或者
\begin{verbatim}
\vskip\MyCount\baselineskip
\end{verbatim}

%The \gr{registerdef} commands are used in plain \TeX\ macros
%\cs{newcount} et cetera that allocate an unused register;
%after
%\begin{verbatim}
%\newcount\MyCount
%\end{verbatim}
%\cs{MyCount} can be used
%exactly as in the above two examples.
\gr{registerdef} 命令用在 \cs{newcount} 等 plain \TeX\ 宏中,
它能分配一个未使用的寄存器;在以下命令之后
\begin{verbatim}
\newcount\MyCount
\end{verbatim}
\cs{MyCount} 就可以完全按照上述两个例子那样使用。

%%\spoint \cs{box}, \cs{fam}, \cs{write}, \cs{read}, \cs{insert}
%\subsection{\cs{box}, \cs{fam}, \cs{write}, \cs{read}, \cs{insert}}
%\spoint \cs{box}, \cs{fam}, \cs{write}, \cs{read}, \cs{insert}
\subsection{\cs{box}, \cs{fam}, \cs{write}, \cs{read}, \cs{insert}}

%For these registers there exists no  \gr{registerdef} command in \TeX,
%so \cs{chardef} is used to allocate box registers
%in the corresponding plain \TeX\ macros \cs{newbox}, for instance.
For these registers there exists no  \gr{registerdef} command in \TeX,
so \cs{chardef} is used to allocate box registers
in the corresponding plain \TeX\ macros \cs{newbox}, for instance.

%The fact that \cs{chardef} is used implies that the
%defined control sequence does not stand for the register itself,
%but only for its number. Thus after
%\begin{verbatim}
%\newbox\MyBox
%\end{verbatim}
%it is necessary to write
%\begin{verbatim}
%\box\MyBox
%\end{verbatim} 
%Leaving out the \cs{box} means that the character
%in the current font with number
%\cs{MyBox} is typeset. The \cs{chardef} command
%is treated further in Chapter~\ref{char}.
The fact that \cs{chardef} is used implies that the
defined control sequence does not stand for the register itself,
but only for its number. Thus after
\begin{verbatim}
\newbox\MyBox
\end{verbatim}
it is necessary to write
\begin{verbatim}
\box\MyBox
\end{verbatim} 
Leaving out the \cs{box} means that the character
in the current font with number
\cs{MyBox} is typeset. The \cs{chardef} command
is treated further in Chapter~\ref{char}.

%\index{registers!allocation of|)}
\index{registers!allocation of|)}

%\section{Ground rules for macro writers}
\section{Ground rules for macro writers}

%The \cs{new...} macros of plain \TeX\ have been designed
%to form a foundation for macro packages, such that
%several of such packages can operate without collisions
%in the same run of \TeX. In appendix~B of \TeXbook\
%Knuth formulates some ground rules that macro writers should
%adhere to.
%\begin{enumerate}
%\item The \cs{new...} macros do not allocate registers
%with numbers~0--9. These can therefore be used as `scratch'
%registers. However, as any macro family can use them,
%no assumption can be made about the permanency of their
%contents. Results that are to be passed from one call to
%another should reside in specifically allocated registers.
The \cs{new...} macros of plain \TeX\ have been designed
to form a foundation for macro packages, such that
several of such packages can operate without collisions
in the same run of \TeX. In appendix~B of \TeXbook\
Knuth formulates some ground rules that macro writers should
adhere to.
\begin{enumerate}
\item The \cs{new...} macros do not allocate registers
with numbers~0--9. These can therefore be used as `scratch'
registers. However, as any macro family can use them,
no assumption can be made about the permanency of their
contents. Results that are to be passed from one call to
another should reside in specifically allocated registers.

%Note that count registers 0--9 are used for page identification
%in the \n{dvi} file (see Chapter~\ref{TeXcomm}), so no global assignments
%to these should be made.
Note that count registers 0--9 are used for page identification
in the \n{dvi} file (see Chapter~\ref{TeXcomm}), so no global assignments
to these should be made.

%\item \cs{count255}, \cs{dimen255}, and \cs{skip255} are
%also available. This is because inserts are
%allocated from 254 downward  and, together with an insertion box,
%a count, dimen, and skip register, 
%all with the same number, are allocated.
%Since \cs{box255} is used by the output routine 
%(see Chapter~\ref{output}),
%the count, dimen, and skip with number~255 are freely available.
\item \cs{count255}, \cs{dimen255}, and \cs{skip255} are
also available. This is because inserts are
allocated from 254 downward  and, together with an insertion box,
a count, dimen, and skip register, 
all with the same number, are allocated.
Since \cs{box255} is used by the output routine 
(see Chapter~\ref{output}),
the count, dimen, and skip with number~255 are freely available.

%\item Assignments to scratch registers~0, 2, 4, 6, 8, and~255
%should be local; assignments to registers~1, 3, 5, 7,~9
%should be \cs{global} (with the exception of the \cs{count}
%registers). This guideline prevents `save
%stack build-up' (see Chapter~\ref{error}).
\item Assignments to scratch registers~0, 2, 4, 6, 8, and~255
should be local; assignments to registers~1, 3, 5, 7,~9
should be \cs{global} (with the exception of the \cs{count}
registers). This guideline prevents `save
stack build-up' (see Chapter~\ref{error}).

%\item Any register can be used inside a group, as \TeX's
%grouping mechanism will restore its value outside
%the group. There are two conditions on this use of
%a register:
%no global assignments should be made to it, and 
%it must not be possible that other macros may be
%activated in that group that perform global assignments
%to that register.
\item Any register can be used inside a group, as \TeX's
grouping mechanism will restore its value outside
the group. There are two conditions on this use of
a register:
no global assignments should be made to it, and 
it must not be possible that other macros may be
activated in that group that perform global assignments
to that register.

%\item Registers that are used over longer periods of time,
%or that have to survive in between calls of different
%macros, should be allocated by \cs{new...}.
%\end{enumerate}
\item Registers that are used over longer periods of time,
or that have to survive in between calls of different
macros, should be allocated by \cs{new...}.
\end{enumerate}

%\endofchapter
%%%%% end of input file [alloc]
\endofchapter
%%%% end of input file [alloc]

\end{document}
