% -*- coding: utf-8 -*-
% Translated by inkydragon@github
% Date of translated: 2018-05-16
\documentclass{book}

\input{preamble}
\setcounter{chapter}{5}

\begin{document}

%\chapter{Horizontal and Vertical Mode}\label{hvmode}
%\label{chap:hvmode}
\chapter{水平模式和竖直模式}\label{hvmode}
\label{chap:hvmode}

%At any point in its processing \TeX\ is in some \indexterm{mode}.
%There are six modes, divided in three categories:
%\begin{enumerate}
%\item horizontal mode and restricted horizontal mode,
%\item vertical mode and internal vertical mode, and
%\item math mode and display math mode.
%\end{enumerate}
%The math modes will be treated elsewhere (see page~\pageref{math:modes}).
%Here we shall look
%at the horizontal and vertical modes, the kinds of objects
%that can occur in the corresponding lists, and the
%commands that are exclusive for one mode or the other. 
在处理文件时,\TeX\ 总处于某种\indexterm{模式}。
一共有六种模式,可划分为三类:
\begin{enumerate}
\item 水平模式和受限水平模式,
\item 竖直模式和内部竖直模式,
\item 数学模式和数学显示模式。
\end{enumerate}
数学模式将会在其他章节详细介绍(见 \pageref{math:modes} 页)。
这里我们首先关注水平和竖直模式,对象的类型会在相应的列表中出现,
并且每个模式的命令都是专用的。

%\label{cschap:vadjust}\label{cschap:showlists}
%\begin{inventory}
%\item [\cs{ifhmode}] 
%      Test whether the current mode is (possibly restricted) horizontal mode.
\label{cschap:vadjust}\label{cschap:showlists}
\begin{inventory}
\item [\cs{ifhmode}] 
      用于测试当前模式是否为(受限)水平模式。

%\item [\cs{ifvmode}] 
%      Test whether the current mode is (possibly internal) vertical mode.
\item [\cs{ifvmode}] 
      用于测试当前模式是否为(内部)竖直模式。

%\item [\cs{ifinner}] 
%      Test whether the current mode is an internal mode.
\item [\cs{ifinner}] 
      用于测试当前模式是否为内部模式。

%\item [\cs{vadjust}] 
%      Specify vertical material for the enclosing vertical list
%      while in horizontal mode.
\item [\cs{vadjust}] 
      指定用于在水平模式中闭合竖直列表的指定竖直符号。

%\item [\cs{showlists}] 
%      Write to the log file the contents of the partial lists 
%      currently being built in all modes.
%\end{inventory}
\item [\cs{showlists}] 
      将目前所有模式正在构建的部分列表的内容写入日志。
\end{inventory}

%%\point Horizontal and vertical mode
%\section{Horizontal and vertical mode}
%\point Horizontal and vertical mode
\section{水平模式和竖直模式}

%When not typesetting mathematics, \TeX\ is in horizontal
%or vertical mode, building horizontal or vertical lists
%respectively. Horizontal mode is typically used to
%make lines of text; vertical mode is typically used
%to stack the lines of a paragraph on top of each other.
%Note that
%these modes
%are different from the internal states of \TeX's input processor
%(see page~\pageref{input:states}).
当不需要处理数学排版时,\TeX\ 处于水平模式或竖直模式中,
分别构建水平列表或竖直列表。
水平模式一般用来生成一行行的文字;
竖直模式则用来将一行行文字组成的段落堆到其他段落上。
注意!这些模式和 \TeX\ 输入处理器的内部状态
(见 \pageref{input:states} 页)并不相同。

%%\spoint Horizontal mode
%\subsection{Horizontal mode}
%\spoint Horizontal mode
\subsection{水平模式}

%The main activity in \indextermbus{horizontal}{mode} is building lines of text.
%Text on the page and text in a \cs{vbox} or \cs{vtop} is built in
%horizontal mode (this might be called `paragraph mode');
%if the text is in an \cs{hbox} there is only one line
%of text, and the corresponding mode is the restricted
%horizontal mode.
\cindextermbus{水平}{模式}的主要用来构建一行行的文字。
页面上的文字和 \cs{vbox} 或 \cs{vtop} 中的文字在水平模式中构建
(这也许叫做“段落模式”);
如果 \cs{hbox} 中仅仅有一行文字,则对应的模式为受限水平模式。

%In horizontal mode all material is added to a horizontal list.
%If this list is built in unrestricted horizontal mode, it
%will later be broken into lines and added to the surrounding vertical list.
在水平模式中,所有的材料都添加进了水平列表。
如果列表在非受限水平模式下构建,稍后它会被分解成行,
然后添加进周围的竖直列表中。

%Each element of a \indextermbus{horizontal}{list} is one of the following:
%\begin{itemize} \item a box (a character, ligature, \cs{vrule},
%or a \gr{box}),
%\item a discretionary break,
%\item a whatsit (see Chapter~\ref{io}),
%\item vertical material enclosed in \cs{mark},
%\cs{vadjust}, or \cs{insert},
%\item 
%\mdqon
%glue or leaders, a kern, a penalty, or a math-on/""off item.
%\mdqoff
%\end{itemize}
%The items in the last point are all discardable.
%\emph{Discardable items}\index{discardable items}
%are called that, because they disappear in
%a break. Breaking of horizontal
%lists is discussed in Chapter~\ref{line:break}.
\cindextermbus{水平}{列表}的每一个元素都是以下各项之一:
\begin{itemize}
  \item 一个盒子(一个字符、合字、\cs{vrule} 或一个 \gr{box}),
  \item 一个软换行
  \item 一个延迟操作(见第 \ref{io} 章)
  \item 由 \cs{mark}, \cs{vadjust} 或 \cs{insert} 闭合的竖直材料,
  \item 
\mdqon
活动铅空或目录连接符、铅空、断行惩罚或数学模式开/""关符号
\mdqoff
\end{itemize}
最后一点的项目都是可丢弃的。
之所以称之为可丢弃项目
(\emph{Discardable items}\index{discardable items/可丢弃项目}),
是因为在列表分解之后它们就消失了。
水平列表的分解会在 \ref{line:break} 章讨论。

%\subsection{Vertical mode}
\subsection{竖直模式}

%\emph{Vertical mode}\index{mode!vertical}
%can be used to stack items on top of one another.
%Most of the time, these items are boxes 
%containing the lines of paragraphs.
\cindextermbus{竖直}{模式}(\emph{Vertical mode})
用于将一个项目堆到另一个上。
大多数时候,这些项目是包含段落行的盒子。

%Stacking material can take place inside a 
%vertical box, but the
%items that are stacked can also 
%appear by themselves on the page. In the latter case
%\TeX\ is in vertical mode; in the former case, inside a
%vertical box, \TeX\ operates in internal vertical mode.
堆叠材料可以在竖直盒子内部进行,但这些堆起来的项目也可以在页面上单独出现。
前一种情况位于一个竖直盒子内部,
\TeX\ 会在内部竖直模式下操作;而后一种情况下 \TeX\ 将处于竖直模式中。

%In vertical mode all material is added to a vertical list.
%If this list is built in external vertical mode, it
%will later be broken when pages are formed.
在竖直模式中所有的材料都会加进竖直列表里。
如果这个列表在内部竖直模式中构建,在页面生成时,它会被分解。

%Each element of a \indextermbus{vertical}{list} is one of the following:
%\begin{itemize}
%\item a box (a horizontal or vertical box or an \cs{hrule}),
%\item a whatsit,
%\item a mark,
%\item glue or leaders, a kern, or a penalty.
%\end{itemize}
%The items in the last point are all discardable.
%Breaking of vertical lists
%is discussed in Chapter~\ref{page:break}.
\cindextermbus{竖直}{列表}的每一个元素都是以下各项之一:
\begin{itemize}
  \item 一个盒子(水平或竖直盒子或一个 \cs{hrule})
  \item 一个延迟操作,
  \item 一个标记,
  \item 活动铅空或目录连接符、铅空或断行惩罚。
\end{itemize}
列表最后一点中的项目都是可丢弃的。
竖直列表的断行将在第 \ref{page:break} 章讨论。

%There are a few exceptional conditions at the beginning
%of  a vertical list: the value of \cs{prevdepth} is set
%to \n{-1000pt}. Furthermore, no \cs{parskip} glue is added
%at the top of an internal vertical list; 
%at the top of the main vertical list (the top of the
%`current page') no glue or other discardable items
%are added, and \cs{topskip} glue is added when the
%first box is placed on this list
%(see Chapters \ref{page:shape} and~\ref{page:break}).
在竖直列表的开始处有一些特殊条件需要注意:
\cs{prevdepth} 的值被设为 \n{-1000pt}。
此外,在内部竖直列表的顶端没有附加 \cs{parskip} 的可变铅空。
在主竖直列表的顶端(即“当前页”的顶端)没有附加可变铅空或其他可丢弃的项目,
\cs{topskip} 的可变铅空会在第一个盒子加入列表时加入
(见第 \ref{page:shape} 和 \ref{page:break} 章)。

%%\point Horizontal and vertical commands
%\section{Horizontal and vertical commands}
%\point Horizontal and vertical commands
\section{水平和竖直命令}

%Some commands are so intrinsically horizontal or vertical
%in nature that they force \TeX\ to go into that mode, if
%possible. A~command that forces \TeX\ into horizontal mode
%is called a \gr{horizontal command}; similarly a command that
%forces \TeX\ into vertical mode is called a
%\gr{vertical command}.
有些命令本质上是“内在”水平或竖直的,如果有可能,它们会强制 \TeX\ 进入该模式。
能强制 \TeX\ 进入水平模式的命令称为水平命令 \gr{horizontal command};
类似的能强制 \TeX\ 进入竖直模式的命令称作竖直命令 \gr{vertical command}。

%However, not all transitions are possible:
%\TeX\ can switch from both vertical modes to 
%(unrestricted) horizontal mode and back
%through horizontal and vertical commands, but no transitions
%to or from restricted horizontal mode are possible
%(other than by enclosing horizontal boxes in vertical boxes or
%the other way around).
%A~vertical command in restricted horizontal mode thus gives
%an error; the \cs{par} command in restricted horizontal mode
%has no effect.
但是,并非所有的模式转换都是可行的:
\TeX\ 能从两种竖直模式切换到(无限制的)水平模式,并可以通过水平和竖直命令切换回去。
但转入或转出受限水平模式的转换均是不可行的(除了在竖直盒子内闭合水平盒子或使用其他的方式)。
在受限水平模式内使用竖直命令会导致错误;
受限水平模式内,\cs{par} 命令没有效果。

%The \indextermbus{horizontal}{commands} are the following:
%\label{h:com:list}
%\begin{itemize}
%\item any \gr{letter}, \gr{otherchar}, \cs{char}, 
%a control sequence defined by \cs{chardef}, or \cs{noboundary};
%\item \cs{accent}, \cs{discretionary}, the discretionary
%hyphen~\verb|\-| and control space~\verb|\|\n{\char32};
%\item \cs{unhbox} and \cs{unhcopy};
%\item \cs{vrule} and the
%\gr{horizontal skip} commands
%\cs{hskip}, \cs{hfil}, \cs{hfill}, \cs{hss}, and \cs{hfilneg};
%\item \cs{valign};
%\item math shift (\n\$).
%\end{itemize}
以下是\cindextermbus{水平}{命令}(\emph{horizontal commands}):
\label{h:com:list}
\begin{itemize}
\item 任何 \gr{letter}, \gr{otherchar}, \cs{char},
由 \cs{chardef} 或 \cs{noboundary} 定义的控制序列;
\item \cs{accent}, \cs{discretionary}, 
软连词符 \verb|\-| 和 控制空格 \verb|\|\n{\char32};
\item \cs{unhbox} 和 \cs{unhcopy};
\item \cs{vrule} 和 \gr{horizontal skip} 命令
\cs{hskip}, \cs{hfil}, \cs{hfill}, \cs{hss}, 和 \cs{hfilneg};
\item \cs{valign};
\item 数学转换 (\n\$)。
\end{itemize}

%The \indextermbus{vertical}{commands} are the following:
%\label{v:com:list}
%\begin{itemize}
%\item \cs{unvbox} and \cs{unvcopy};
%\item \cs{hrule} and the \gr{vertical skip} commands
% \cs{vskip}, \cs{vfil}, \cs{vfill}, \cs{vss}, and \cs{vfilneg};
%\item \cs{halign};
%\item \cs{end} and \cs{dump}.
%\end{itemize}
%Note that the vertical commands do not include \cs{par};
%nor are \cs{indent} and \cs{noindent} horizontal commands.
以下是\cindextermbus{竖直}{命令}(\emph{vertical commands}):
\label{v:com:list}
\begin{itemize}
\item \cs{unvbox} 和 \cs{unvcopy};
\item \cs{hrule} 和 \gr{vertical skip} 命令
 \cs{vskip}, \cs{vfil}, \cs{vfill}, \cs{vss} 和 \cs{vfilneg};
\item \cs{halign};
\item \cs{end} 和 \cs{dump}.
\end{itemize}
注意!竖直命令不包含 \cs{par};
\cs{indent} 和 \cs{noindent} 也不是水平命令。

%The connection between boxes and modes is explored below;
%see Chapter~\ref{rules} for more on the connection between
%rules and modes.
下方探讨了盒子和模式的联系;
关于规则和模式的联系,详见第 \ref{rules} 章。

%\section{The internal modes}
\section{内部模式}

%The \indextermbus{restricted horizontal}{mode} and 
%\indextermbus{internal vertical}{mode}
%are those variants of horizontal mode and vertical mode
%that hold inside an \cs{hbox} and \cs{vbox} (or \cs{vtop}
%or \cs{vcenter}) respectively.
%However, restricted horizontal mode is rather more
%restricted in nature than internal vertical mode.
%The third internal mode is non-display math mode
%(see Chapter~\ref{math}).
\cindextermbus{受限水平}{模式}是 \cs{hbox} 中的水平模式的变种;
而\cindextermbus{内部竖直}{模式}是 \cs{vbox} (或 \cs{vtop}, \cs{vcenter}) 中竖直模式的变种。
本质上说,受限水平模式相对于内部竖直模式限制更多。
第三种内部模式是非显示的数学模式(见 \ref{math} 章)

%%\spoint Restricted horizontal mode
%\subsection{Restricted horizontal mode}
%\spoint Restricted horizontal mode
\subsection{受限水平模式}

%The main difference between restricted horizontal mode,
%the mode in an \cs{hbox}, and unrestricted horizontal mode,
%the mode in which paragraphs in vertical boxes
%and on the page are built,
%is that you cannot break out of restricted horizontal mode: 
%\cs{par}~does nothing in this mode. 
%Furthermore, a~\gram{vertical command} in restricted horizontal
%mode gives an error. 
%In unrestricted horizontal mode it would cause a
%\cs{par} token to be inserted and vertical mode to be entered
%(see also Chapter~\ref{par:end}).
对于水平模式,受限(\cs{hbox}的模式)与不受限(正在构建页面和竖直盒子中段落的模式)的最大区别是:你不能退出受限水平模式,而且 \cs{par} 在受限模式中不起作用。
此外,受限水平模式中的 \gram{vertical command} 会导致错误。
再非受限水平模式中,这样做会插入一个 \cs{par} 记号,并进入竖直模式(详见第 \ref{par:end} 章)。

%%\spoint Internal vertical mode
%\subsection{Internal vertical mode}
%\spoint Internal vertical mode
\subsection{内部竖直模式}

%Internal vertical mode, the vertical mode inside
%a~\cs{vbox}, is a lot like external vertical
%mode, the mode in which pages are built.
%A~\gram{horizontal command} in internal vertical mode,
%for instance, is perfectly valid:
%\TeX\ then starts building a paragraph in
%unrestricted horizontal mode.
内部竖直模式即 \cs{vbox} 内的竖直模式,很像构建页面时使用的外部竖直模式。
内部竖直模式内的 \gram{horizontal command} 是完全有效的,例如:\TeX\ 开始以非受限水平模式构建段落。

%One difference is that the commands
%\cs{unskip} and \cs{unkern} have no effect
%in external vertical mode, and
%\cs{lastbox} is always empty in external vertical mode.
%See further pages \pageref{lastbox} and~\pageref{unskip}.
内部和外部竖直模式的一个区别是:在内部竖直模式中,\cs{unskip} 和 \cs{unkern} 命令没有效果,并且 \cs{lastbox}s 总是空的。
详见第 \pageref{lastbox} 和 \pageref{unskip}。

%The entries of alignments (see Chapter~\ref{align}) are 
%processed in internal modes: restricted horizontal mode
%for the entries of an \cs{halign}, and internal vertical
%mode for the entries of a~\cs{valign}.
%The material in \cs{vadjust}   and \cs{insert} items
%is also processed in internal vertical mode; furthermore,
%\TeX\ enters this mode when processing the \cs{output} token list.
输入的对齐在内部模式中处理(见第 \ref{align} 章)。
\cs{halign} 的项使用受限水平模式,而 \cs{valign} 的项使用内部竖直模式。
\cs{vadjust} 的材料和 \cs{insert} 的项也在内部竖直模式内处理;此外,\TeX\ 在处理 \cs{output} 记号序列时也会进入这个模式。

%The commands \cs{end} and \cs{dump} (the latter exists only in \IniTeX)
%are not allowed in
%internal vertical mode; furthermore, \cs{dump} is not allowed
%inside a group (see Chapter~\ref{TeXcomm}).
\cs{end} 和 \cs{dump} 命令不允许出现在内部竖直模式中(后者仅存在于 \IniTeX 中);
此外 \cs{dump} 不允许在组内使用(见底 \ref{TeXcomm} 章)。

%%\point[hvbox]  Boxes and modes
%\section{Boxes and modes}
%\label{hvbox}
%\point[hvbox]  Boxes and modes
\section{盒子与模式}
\label{hvbox}

%There are horizontal and vertical boxes, and there is
%horizontal and vertical mode. Not surprisingly, there is
%a connection between the boxes and the modes.
%One can ask about this connection in two ways.
\TeX\ 有水平盒子和竖直盒子,也有水平模式和竖直模式。
毫不奇怪,盒子与模式之间肯定有关系。你可能会产生以下疑问。

%%\spoint What box do you use in what mode?
%\subsection{What box do you use in what mode?}
%\spoint What box do you use in what mode?
\subsection{某种模式下应该用哪种盒子?}

%This is the wrong question. Both horizontal  and vertical boxes
%can be used in both horizontal and vertical mode. 
%Their placement is determined by the prevailing mode at that moment.
这个问法不对。水平盒子和竖直盒子均可以用在水平模式和竖直模式中。
它们的放置的位置取决于那一刻的状态。

%%\spoint What mode holds in what box?
%\subsection{What mode holds in what box?}
%\spoint What mode holds in what box?
\subsection{某个盒子里应该用哪种模式?}

%This is the right question.
%When an \cs{hbox} starts, \TeX\ is in restricted horizontal
%mode. Thus everything in a horizontal box is lined up horizontally.
这才是正确的问法。当一个 \cs{hbox} 开始时,\TeX\ 处于受限水平模式。
这就是为什么在水平盒子里,所有的东西都呈水平线性排列。

%When a \cs{vbox} is started, \TeX\ is in internal vertical mode.
%Boxes of both kinds and other items are then stacked
%on top of each other.
当一个 \cs{vbox} 开始时,\TeX\ 进入内部竖直模式。
两种模式下的盒子和其它的项目都会一个摞一个的堆叠起来。


%%\spoint Mode-dependent behaviour of boxes
%\subsection{Mode-dependent behaviour of boxes}
%\spoint Mode-dependent behaviour of boxes
\subsection{盒子的模式无关行为}

%Any \gr{box} (see Chapter \ref{boxes} for the full definition)
%can be used in horizontal, vertical, and math mode.
%Unboxing commands, however, are specific for horizontal or vertical mode.
%Both \cs{unhbox} and \cs{unhcopy} are \gr{horizontal command}s,
%so they can make \TeX\ switch from vertical to horizontal
%mode; 
%both \cs{unvbox} and \cs{unvcopy} are \gr{vertical command}s,
%so they can make \TeX\ switch from horizontal to vertical
%mode.
任何盒子 \gr{box} (详细定义见第 \ref{boxes} 章)都能用于水平、竖直和数学模式中。
但是,解包命令对于水平或竖直模式是特定的。
\cs{unhbox} 和 \cs{unhcopy} 都是 \gr{horizontal command},所以它们能使 \TeX\ 从竖直模式切换成水平模式。
\cs{unvbox} 和 \cs{unvcopy} 都是 \gr{vertical command},所以它们能使 \TeX\ 从水平模式切换成竖直模式。

%In horizontal mode the \cs{spacefactor} is set to 1000
%after a box has been placed. In vertical mode the
%\cs{prevdepth} is set to the depth of the box placed.
%Neither statement holds for
%unboxing commands: after an \cs{unhbox} or \cs{unhcopy} the 
%spacefactor is not altered, and after \cs{unvbox} or \cs{unvcopy}
%the \cs{prevdepth} remains unchanged.
%After all, these commands do not add a box,
%but a piece of a~(horizontal or vertical) list.
水平模式中,在放置了一个盒子之后 \cs{spacefactor} 被设定为 1000.
竖直模式中 \cs{prevdepth} 被设定为盒子放置时的深度。
这两个语句都不支持解包命令:
在 \cs{unvbox} 或 \cs{unvcopy} 之后spacefactor的值不变;
并且在 \cs{unhbox} 或 \cs{unhcopy} 之后 \cs{prevdepth} 保持原值。
毕竟,这些命令并不会添加一个新的盒子,只是添加了水平或竖直列表的一部分。

%The operations \cs{raise} and \cs{lower} can only be
%applied to a box in horizontal mode; similarly, \cs{moveleft} and
%\cs{moveright} can only be applied in vertical mode.
\cs{raise} 和 \cs{lower} 操作只能用于水平模式中的盒子。
类似的,\cs{moveleft} 和 \cs{moveright} 只能用于竖直模式中的盒子.


%%\point Modes and glue
%\section{Modes and glue}
%\point Modes and glue
\section{模式和伸缩胶}

%Both in horizontal and vertical mode
%\TeX\ can insert glue items the size of which is
%determined by the preceding object in the list.
在水平模式和竖直模式中,\TeX\ 都能插入一个伸缩胶项,它的大小由列表里之前的对象决定。

%For horizontal mode the amount of glue that is inserted
%for a space token depends on the \cs{spacefactor} of
%the previous object in the list. This is treated
%in Chapter~\ref{space}.
对于水平模式,为一个空格记号插入的伸缩胶的量,由列表里前一个量的 \\ 
\cs{spacefactor} 值决定。这会在第 \ref{space} 章讨论。

%In vertical mode \TeX\ inserts glue to keep boxes at a certain
%distance from each other. This glue is influenced by the
%height of the current item and the depth of the previous one.
%The depth of items is recorded in the \cs{prevdepth} parameter
%(see Chapter~\ref{baseline}).
\TeX\ 在竖直模式中插入伸缩胶,让盒子之间都保持特定的距离。
伸缩胶会受到当前项高度前一项深度的影响。
项目的深度记录在 \cs{prevdepth} 参数中(见第 \ref{baseline} 章)。

%The two quantities \cs{prevdepth} 
%and \cs{spacefactor} 
%use the same internal register of \TeX. Thus the \cs{prevdepth}
%can be used or asked only in vertical mode, and the \cs{spacefactor}
%only in horizontal mode.
\cs{prevdepth} 和 \cs{spacefactor} 这两个量使用了 \TeX\ 中相同的内部寄存器。
因此 \cs{prevdepth} 只能在竖直模式中使用或请求,\cs{spacefactor} 只能用于水平模式中。

%%\point[migrate] Migrating material
%\section{Migrating material}
%\label{migrate}
%\point[migrate] Migrating material
\section{迁移材料}
\label{migrate}

%The three control sequences \cs{insert}, \cs{mark}, and \cs{vadjust}
%can be given in a paragraph 
%(the first two can also occur
%in vertical mode) to specify \indexterm{migrating material}:
%material that will wind up on the
%surrounding vertical list rather than on the current list.
%Note that this need not be 
%the main vertical list: it can be a vertical box
%containing a paragraph of text. In this case a \cs{mark}
%or \cs{insert} command will not reach the page breaking algorithm.
在一段话中,可以用三种控制序列 \cs{insert}, \cs{mark} 和 \cs{vadjust}指定\indexterm{迁移材料} (前两个也可以出现在竖直模式中):
材料会在周围的竖直列表而不是当前列表中结束。
注意这并不一定是主竖直列表:也可以是包含一段文字的竖直盒子。
这种情况下 \cs{mark} 或 \cs{insert} 命令不会触发分页算法。

%When several migrating items are specified in a certain line
%of text, their left-to-right order is preserved when they are
%placed on the surrounding vertical list. These items are placed
%directly after the horizontal box containing the line of text
%in which they were specified: they come before any
%penalty or glue items that are automatically inserted
%(see page~\pageref{between:lines}).
在特定的文本行中指定了多个迁移项目后,当将其放置在周围的竖直列表中时,它们从左至右的顺序会保留下来。
这些项目直接放置在包含指定文本的水平盒子之后:它们会在任何自动生成的惩罚值或伸缩胶项目之前插入(见第 \pageref{between:lines} 页)。

%%\spoint \cs{vadjust}
%\subsection{\cs{vadjust}}
%\spoint \cs{vadjust}
\subsection{\cs{vadjust}}

%The command
%\cstoidx vadjust\par
%\begin{disp}\cs{vadjust}\gr{filler}\lb\gr{vertical mode material}\rb\end{disp}
%is only allowed in horizontal and math modes (but it is
%not a \gr{horizontal command}).
%Vertical mode material specified by \cs{vadjust} is moved from
%the horizontal list in which the command is given
%to the surrounding vertical list, directly after the box
%in which it occurred.
命令 
\cstoidx vadjust\par
\begin{disp}\cs{vadjust}\gr{filler}\lb\gr{vertical mode material}\rb\end{disp}
只允许在水平和数学模式中使用 (但它不是一个\gr{horizontal command})。
由 \cs{vadjust} 指定的竖直模式材料,将从给出该命令的水平列表中移动到周围的竖直列表中,直接放置在它原本出现的盒子之后。

%In the current line
%\vadjust{\setbox0=\hbox{$\bullet$\hskip1em}\ht0=0pt \dp0=0pt \llap{\box0}}
%a \cs{vadjust} item was placed to put the bullet in the margin.
在当前行中,
\vadjust{\setbox0=\hbox{$\bullet$\hskip1em}\ht0=0pt \dp0=0pt \llap{\box0}}
放置了一个 \cs{vadjust} 项目,它在页边中插入了一个点。


%Any vertical material in a \cs{vadjust} item is processed
%in internal vertical mode, even though it will wind up
%on the main vertical list. For instance, the \cs{ifinner}
%test is true in a \cs{vadjust}, and at the start
%\mdqon
%of the vertical material \cs{prevdepth}$=$""\n{-1000pt}.
%\mdqoff
\cs{vadjust} 项目中的任何竖直材料会在内部竖直模式中处理,即使它会结束主竖直列表。
举个例子,在 \cs{vadjust} 中 \cs{ifinner} 测试为真,并且在竖直材料的开头
\mdqon
\cs{prevdepth}$=$""\n{-1000pt}。
\mdqoff

%%\point Testing modes
%\section{Testing modes}
%\point Testing modes
\section{测试模式}

%The three conditionals \cs{ifhmode}, \cs{ifvmode}, and
%\cs{ifinner} can distinguish between the four modes of
%\TeX\ that are not math modes.
%The \cs{ifinner} test is true if \TeX\ is in 
%restricted horizontal mode or internal vertical mode
%(or in non-display math mode).
%Exceptional condition: during a \cs{write} \TeX\
%is in a `no mode' state. The tests \cs{ifhmode},
%\cs{ifvmode}, and \cs{ifmmode} are then all false.
三种条件语句 \cs{ifhmode}, \cs{ifvmode} 和 \cs{ifinner} 可以用来区分非数学模式的四种 \TeX\ 模式。
当 \TeX\ 处于受限水平模式、内部竖直模式或非显示数学模式中时, \cs{ifinner} 测试为真。
有一个例外情况:在 \cs{write} 指令执行的时候, \TeX\ 处于无模式状态。此时 \cs{ifhmode}, \cs{ifvmode} 和 \cs{ifmmode} 测试都为假。

%Inspection of all current lists, including the `recent
%contributions' (see Chapter~\ref{page:break}),
%is possible through the command \csidx{showlists}\label{showlists}.
%This command writes to the log file the contents of all
%lists that are being built at the moment the command is given.
我们可以通过 \csidx{showlists}\label{showlists} 命令检查所有列表的值,包括“最近贡献”(见第 \ref{page:break} 章)。
这个命令会把执行此命令时,所有正在构建的列表的内容写入日志里。

%Consider the example
%\begin{verbatim}
%a\hfil\break b\par 
%c\hfill\break d
%\hbox{e\vbox{f\showlists
%\end{verbatim}
%Here the first paragraph has been broken into two lines, and
%these have been added to the current page. The second paragraph
%has not been concluded or broken into lines.
考虑以下例子
\begin{verbatim}
a\hfil\break b\par 
c\hfill\break d
\hbox{e\vbox{f\showlists
\end{verbatim}
这里第一段被分为两行,这两部分被添加进当前页中。
第二段尚未结束,也没有被分为多行。

%The log file shows the following. \TeX\ was busy
%building a paragraph (starting with an indentation box
%\n{20pt} wide):\begin{verbatim}
%### horizontal mode entered at line 3
%\hbox(0.0+0.0)x20.0
%\tenrm f
%spacefactor 1000
%\end{verbatim}
%This paragraph was inside a vertical box:\begin{verbatim}
%### internal vertical mode entered at line 3
%prevdepth ignored
%\end{verbatim}
%The vertical box was in  a horizontal box, 
%\begin{verbatim}
%### restricted horizontal mode entered at line 3
%\tenrm e
%spacefactor 1000
%\end{verbatim}
%which was part of
%an as-yet unfinished paragraph:\begin{verbatim}
%### horizontal mode entered at line 2
%\hbox(0.0+0.0)x20.0
%\tenrm c
%\glue 0.0 plus 1.0fill
%\penalty -10000
%\tenrm d
%etc.
%spacefactor 1000
%\end{verbatim}
%Note how the infinite glue and the \cs{break} penalty
%are still part of the horizontal list.
以下是日志的记录。
\TeX\ 正在忙于构建一个段落(从一个\n{20pt}宽的缩进盒子开始):
\begin{verbatim}
### horizontal mode entered at line 3
\hbox(0.0+0.0)x20.0
\tenrm f
spacefactor 1000
\end{verbatim}
这个段落处于一个竖直盒子中:
\begin{verbatim}
### internal vertical mode entered at line 3
prevdepth ignored
\end{verbatim}
竖直盒子处于一个水平盒子中, 
\begin{verbatim}
### restricted horizontal mode entered at line 3
\tenrm e
spacefactor 1000
\end{verbatim}
而水平盒子又是另一个未完成段落的一部分:
\begin{verbatim}
### horizontal mode entered at line 2
\hbox(0.0+0.0)x20.0
\tenrm c
\glue 0.0 plus 1.0fill
\penalty -10000
\tenrm d
etc.
spacefactor 1000
\end{verbatim}
注意到无限制的伸缩胶和断行\cs{break}惩罚值仍然是水平列表的一部分。

%Finally, the first paragraph has been broken into lines and 
%added to the current page:\begin{verbatim}
%### vertical mode entered at line 0
%### current page:
%\glue(\topskip) 5.69446
%\hbox(4.30554+0.0)x469.75499, glue set 444.75497fil
%.\hbox(0.0+0.0)x20.0
%.\tenrm a
%.\glue 0.0 plus 1.0fil
%.\penalty -10000
%.\glue(\rightskip) 0.0
%\penalty 300
%\glue(\baselineskip) 5.05556
%\hbox(6.94444+0.0)x469.75499, glue set 464.19943fil
%.\tenrm b
%.\penalty 10000
%.\glue(\parfillskip) 0.0 plus 1.0fil
%.\glue(\rightskip) 0.0
%etc.
%total height 22.0 plus 1.0
% goal height 643.20255
%prevdepth 0.0
%\end{verbatim}
最终,第一个段落被分为行然后加入到当前页中:
\begin{verbatim}
### vertical mode entered at line 0
### current page:
\glue(\topskip) 5.69446
\hbox(4.30554+0.0)x469.75499, glue set 444.75497fil
.\hbox(0.0+0.0)x20.0
.\tenrm a
.\glue 0.0 plus 1.0fil
.\penalty -10000
.\glue(\rightskip) 0.0
\penalty 300
\glue(\baselineskip) 5.05556
\hbox(6.94444+0.0)x469.75499, glue set 464.19943fil
.\tenrm b
.\penalty 10000
.\glue(\parfillskip) 0.0 plus 1.0fil
.\glue(\rightskip) 0.0
etc.
total height 22.0 plus 1.0
 goal height 643.20255
prevdepth 0.0
\end{verbatim}


%\endofchapter
%%%%% end of input file [modes]
\endofchapter
%%%% end of input file [modes]

\end{document}
